\documentclass[paper=a4, 
               fontsize=11pt, 
               DIV=13, 
               BCOR=14.0mm, 
               titlepage=true,
               ngerman]
               {scrbook}
\usepackage[utf8]{inputenc}
\usepackage[ngerman]{babel}
\usepackage{scrpage2}
\usepackage{typearea}
\usepackage{microtype}
\usepackage{float}
\usepackage{scrtime}

\usepackage{amsmath}
\usepackage{amssymb}
\usepackage{amsfonts}
\usepackage{graphicx}
\usepackage{cite}

\usepackage[standard]{ntheorem}
\usepackage{enumitem} 

\newtheorem{thm}{Satz}[chapter]
\numberwithin{equation}{chapter}
\numberwithin{Definition}{chapter}
\numberwithin{definition}{chapter}
\numberwithin{Korollar}{chapter}
\numberwithin{korollar}{chapter}
\numberwithin{Lemma}{chapter}
\numberwithin{lemma}{chapter}
\numberwithin{Proposition}{chapter}
\numberwithin{proposition}{chapter}
\numberwithin{Beispiel}{chapter}
\numberwithin{Bemerkung}{chapter}

% some KOMA stuff
\pagestyle{scrheadings}
\renewcommand*{\chapterpagestyle}{scrheadings} 
\setheadsepline{0.25pt}
\setfootsepline{0.25pt}
\automark[chapter]{section}
\renewcommand{\baselinestretch}{1.2}\normalsize

%\setkomafont{section}{\fontsize{20bp}{8.8bp}\selectfont\bfseries}
\renewcommand*{\headfont}{%
\normalfont}%\sffamily}%\bfseries}

% kommutative Diagramme
\usepackage[arrow, matrix, curve]{xy}

% custom commands
\newcommand{\rtrace}[1]{\mbox{ }^R\operatorname{Tr} \left( #1 \right)}
\newcommand{\rint}{\mbox{ }^R \int\limits_M}
\newcommand{\heatop}[1]{e^{-#1\Delta_\tau}}
\newcommand{\heatopp}[2]{e^{-#1\Delta_{#2}}}

\newcommand{\FP}[1]{\underset{#1=0}{\operatorname{FP}}}
\newcommand{\wheat}{\omega'(\tau)\left( \heatop{t} - \mathcal{P} \right) }
\newcommand{\wheatt}{\omega'(\tau) \heatop{t} }
\newcommand{\del}[1]{\partial_{#1}}
\newcommand{\duhamel}[1]{\int_0^t \heatop{(t-s)} #1 \heatop{s} \mathrm{ds}}
\newcommand{\tr}[1]{\operatorname{Tr}\left( #1 \right)}
\newcommand{\trzero}[1]{\operatorname{Tr}_{0}\left( #1 \right)}
\newcommand{\kernel}[1]{\left. #1 \right|_{\operatorname{diag}}}
\newcommand{\diag}{\operatorname{diag}}
\newcommand{\lpspace}{L^2(M,g)}
\newcommand{\psiheat}[2]{\Psi^{#1,#2}_{e,\text{Heat}}(M,E)}
\newcommand{\heatkernel}[1]{\left. \mathcal{K}_{\Delta_{#1}} \right|_{\operatorname{diag}}}
\newcommand{\vol}{\operatorname{vol}}
\newcommand{\funnels}{\bigcup_{i=1}^{n_F} \mathcal{F}_i}
\newcommand{\morespace}{\\ \mbox{} \\}
\newcommand{\phg}{\mathcal{A}^{\ast}_{\operatorname{phg}}(M)}
\newcommand{\phgX}{\mathcal{A}^{\ast}_{\operatorname{phg}}(X)}

\newcommand{\area}{\operatorname{Area}(g_\tau)}
\newcommand{\invarea}{\frac{1}{\area}}
\newcommand{\brackets}[2]{\langle #1,#2 \rangle}
\newcommand{\polyakov}{ \log \frac{\det \Delta_1}{\det \Delta_0} = -\frac{1}{12 \pi} \int_M \{ |\nabla \varphi|_0^2 + 2 \varphi K_0\} \mathrm{dvol_{g_0}}}

\newcommand{\var}[2]{\left. \mathrm{\frac{d}{d#1}}\right|_{#1 = 0} #2 }
\newcommand{\genvar}[3]{\left. \mathrm{\frac{d}{d#1}}\right|_{#1 = #2} #3 }
\newcommand{\vardel}[2]{\left. \del{#1}\right|_{#1 = 0} #2 }
\newcommand{\proj}[1]{\mathcal{P}_{\ker \Delta_{#1}}}
\newcommand{\der}[1]{\mathrm{\frac{d}{d#1}}}
\renewcommand{\Re}{\operatorname{Re}}

\allowdisplaybreaks[1]
\newcommand{\boxeqn}[1]{
\[
\fbox{
\addtolength{\linewidth}{-2\fboxsep}%
\addtolength{\linewidth}{-2\fboxrule}%
\begin{minipage}{\linewidth}
\begin{align*}
#1
\end{align*}
\end{minipage}
}
\]
}

%opening
\title{Hier steht das Thema}
\subject{Diplomarbeit}
\author{Robin Neumann}
% \publishers

\begin{document}
%\textbf{timestamp:} This file was compiled \today, \thistime
\thispagestyle{empty}

\hspace{20cm}
\vspace{0cm}
\begin{figure}
[H] %\hspace{11cm}

\begin{center}
  \vspace{0.5 cm}
  \huge{\bf Was ich schon immer mal sagen wollte} \\ 
  \vspace{1.5cm}
  \LARGE  Diplomarbeit \\ .
  \vspace{1cm}
  
\begin{center}
%\includegraphics[width=3.2 cm]{HU_Logo} % Logo of your university goes here
\vspace{1.0cm} 
\end{center}
  \vspace{2cm}
  {\large
    \bf{
      \scshape
      Humboldt-Universit\"at zu Berlin \\
      Mathematisch-Naturwissenschaftliche Fakult\"at II \\
      Institut f\"ur Mathematik\\
    }
  } 
  % \normalfont
\end{center}
\end{figure}

\vspace {2.1 cm}% gegebenenfalls kleiner, falls der Titel der Arbeit sehr lang sein sollte
%{3.2 cm} bei Verwendung von scrreprt, gegebenenfalls kleiner, falls der Titel der Arbeit sehr lang sein sollte
{\large
  \begin{tabular}{llll}
    eingereicht von:    & Max Musterstudent && \\ % Bitte Vor- und Nachnamen anstelle der Punkte eintragen.
    geboren am:         & 01.02.2010 && \\
    in:                 & Berlin && \\
    &&&\\
    Gutachter(innen): & Prof. Dr. A. Beispielprof && \\
		      & Prof. Dr. B. Beispielprof&& \\% Bitte Namen der Gutachter(innen) anstelle der Punkte eintragen
				 % bei zwei männlichen Gutachtern kann das (innen) weggestrichen werden
    &&&\\
    eingereicht am:     & \dots\dots  \hspace{3cm} verteidigt am: & \dots\dots \\ % Bitte lassen Sie
  \end{tabular}
} 
\tableofcontents
\chapter{Einführung}
Hier steht ein wenig einführender Text, der den folgenden Satz motiviert:
\begin{thm}[\cite{Polyakov1981}]
Sei $M$ eine geschlossene Fläche, $g_0$ eine Riemannsche Metrik auf $M$ und $g_1 = e^{2\varphi}g_0$ für ein $\varphi \in C^{\infty}(M)$. Es gelte
weiterhin $\operatorname{vol}(M,g_0) = \operatorname{vol}(M,g_1)$. $\Delta_0$ bezeichne den Laplace-Beltrami-Operator
zu $g_0$ und $\Delta_1$ den Laplace-Beltrami-Operator zu $g_1$. Dann gilt:
\begin{equation}
\label{eq:polyakov}
\polyakov.
\end{equation}
\end{thm}
\begin{Beweis}
\dots
\flushright$\square$ 
\end{Beweis}

\section{Dies ist ein Unterabschnitt}
Hier stehen nützliche Dinge.
\nocite{*} % Sagt: Zeige alle Quellen aus der Datei, auch die, die nicht direkt zitiert wurden
\bibliographystyle{amsalpha}
\bibliography{literatur}
\end{document}
